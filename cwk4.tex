\documentclass{article}
\usepackage{amsmath}
\usepackage{amssymb}

\begin{document}

\textbf{Claim}: If $G \cong H$, then $\overline{G} \cong \overline{H}$, where $\overline{G}$ and $\overline{H}$ are the complements of $G$ and $H$ respectively.

\textbf{Proof}: Let $G = (V_G, E_G)$ and $H = (V_H, E_H)$ be simple graphs. Assume $G \cong H$, meaning there exists an isomorphism $f: V_G \rightarrow V_H$ such that for any two vertices $u, v \in V_G$, $u$ and $v$ are adjacent in $G$ if and only if $f(u)$ and $f(v)$ are adjacent in $H$.

Now, let's define a function $g: V_G \rightarrow V_H$ such that $g(u) = f(u)$ for all $u \in V_G$. It's clear that $g$ is also a bijection.

We will show that $g$ is an isomorphism between the complements $\overline{G}$ and $\overline{H}$.

\textbf{Vertex Correspondence}: Since $g$ is a bijection, it establishes a one-to-one correspondence between the vertices of $\overline{G}$ and $\overline{H}$.

\textbf{Edge Correspondence}: Let $(u, v)$ be an edge in $\overline{G}$, then $(u, v)$ is not an edge in $G$. Since $G \cong H$, $f(u)$ and $f(v)$ are not adjacent in $H$. Therefore, $(f(u), f(v))$ is an edge in $\overline{H}$. This implies that $(g(u), g(v)) = (f(u), f(v))$ is an edge in $\overline{H}$. Conversely, if $(u, v)$ is not an edge in $\overline{G}$, then $(f(u), f(v))$ is not an edge in $\overline{H}$.

\textbf{Isomorphism}: Since $g$ preserves adjacency, it is an isomorphism between $\overline{G}$ and $\overline{H}$.

Therefore, if $G \cong H$, then $\overline{G} \cong \overline{H}$.

\textbf{Q2}

(a).To find the total number of edges, we can use the formula for the number of edges in a complete graph:

Number of edges in \[K_n =  \frac{n(n-1)}{2} \]

(b).
In the case of Kn, it is bipartite if and only if n is even. When n is odd, Kn cannot be bipartite because it would require an odd number of vertices in each partition, which is not possible. Therefore, Kn is bipartite when n is \textbf{even}.

(c).
For every vertex in X there are n vertex in Y that are adjacent to it, so
\[ k_{m,n} = mn \]

(d).
According to the Euler's Theorem, A connected graph with at least one edge has an Euler cycle if and only if it has
no vertices of odd degree.  For the complete bipartite graph Km,n to be Eulerian, both m and n must be even.

\textbf{Q3}

This question can be proved by contradiction
Suppose $ |X| \neq |Y| $
Without loss of generality, assume $|X| < |Y|$.
Since G is r-regular, the total number of edges in G is $|E| = r|X|$.
But each edge connects one vertex from X to one vertex from Y, so $|E| = r|X|$ must be equal to $|Y|$ since $|X| < |Y|$.
This implies that there are not enough vertices in X to accommodate all the edges.
Therefore, there must be some vertex in Y that does not have a neighbor in X.
This contradicts the assumption that G is bipartite, as every edge connects a vertex from X to a vertex from Y.
Hence, our assumption that $|X| \neq |Y|$ is false
\end{document}
