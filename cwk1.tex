\documentclass{article}
\usepackage{amsmath}

\begin{document}
1. 

\underline{Constraint (a) :}

For constraint (a), where both the bride and the groom must be in the picture, we need to choose 3 people from the remaining 8 (excluding the bride and the groom) to join them. This can be calculated using combinations.

The number of ways to choose 3 people from 8 is $ C(8, 3) $. Then, as the bride and groom are fixed, we arrange these 5 people in a row, which is $ P(5, 5) $.

So, the total number of arrangements for constraint (a) is $ C(8, 3) \times P(5, 5) $.

Calculating:
\[ C(8, 3) = \frac{8!}{3!(8-3)!} = \frac{8!}{3!5!} = 56 \]
\[ P(5, 5) = 5! = 120 \]

\[ \text{Total arrangements for constraint (a)} = C(8, 3) \times P(5, 5) = 56 \times 120 = 6720 \]

\underline{Constraint (b) :}

For constraint (b), where the bride must be next to the groom, we consider them as a single entity $ BG $. Then, we have 4 entities to arrange: $ BG, P_1, P_2, $ and $ P_3 $. This can be calculated using permutations.

The number of ways to arrange the bride and groom as a pair is $ P(2, 2) $. Then, we choose 3 people from the remaining 8 (excluding the bride and the groom) to join them. This can be calculated using combinations, $ C(8, 3) $. Finally, we arrange the remaining 4 entities, which is $ P(4, 4) $.

So, the total number of arrangements for constraint (b) is $ P(2, 2) \times C(8, 3) \times P(4, 4) $.

Calculating:
\[ P(2, 2) = 2! = 2 \]
\[ C(8, 3) = \frac{8!}{3!(8-3)!} = \frac{8!}{3!5!} = 56 \]
\[ P(4, 4) = 4! = 24 \]

\[ \text{Total arrangements for constraint (b)} = P(2, 2) \times C(8, 3) \times P(4, 4) = 2 \times 56 \times 24 = 2688 \]

2. To solve this problem, we need to consider the circular arrangement of 10 people around a table. 

Given:
\begin{itemize}
    \item Two seatings are considered the same when everyone has the same immediate left and immediate right neighbor.
\end{itemize}

In a circular arrangement, the number of ways to arrange n distinct objects is (n-1)!. This is because we can fix one person and arrange the others relative to them, which eliminates the identical arrangements that come from rotations.

To solve this, we can fix one person in a seat, and then arrange the remaining 9 people around the table. Since the table is circular, the position of the fixed person doesn't matter.
So the number of ways to arrange 9 people in a row is \( P(9, 9) \) (permutations of 9 objects).

Calculating:
\[ P(9, 9) = 9! = 362,880 \]

\textbf{3.}
\textbf{This problem can be divided into two parts:}

\textbf{Poker hands that do not contain a King}: There are 4 Kings in a deck of 52 cards. So, there are 52 - 4 = 48 cards that are not Kings. The number of ways to choose 5 cards from these 48 is given by the combination formula “48 choose 5”, 
denoted as $C(48, 5)$ .


\textbf{Poker hands that contain exactly one suit}: Since there are 4 suits and each suit has 13 cards, the number of ways to choose 5 cards from one suit is given by “13 choose 5”, denoted as $C(13, 5)$. Since there are 4 suits, the total number of such hands is 
$4 \times C(13, 5)$.


However, there is an overlap between these two categories: hands that contain exactly one suit and do not contain a King. These are counted in both of the above categories, so we need to subtract them out once. The number of such hands is $C(12, 5)$, since we are choosing 5 cards from the 12 non-King cards in one suit.
So, the total number of poker hands that do not contain a King or contain exactly one suit is:

\[ C(48, 5) + 4 \times C(13, 5) - 4 \times C(12, 5) = 1,714,284 \]
\end{document}
